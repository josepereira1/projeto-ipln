\documentclass[11pt,a4paper]{report}%especifica o tipo de documento que tenciona escrever: carta, artigo, relatório... neste caso é um relatório
% [11pt,a4paper] Define o tamanho principal das letras do documento. caso não especifique uma delas, é assumido 10pt
% a4paper -- Define o tamanho do papel.
\usepackage{float}
\usepackage[portuges]{babel}%Babel -- irá activar automaticamente as regras apropriadas de hifenização para a língua todo o
                                   %-- o texto gerado é automaticamente traduzido para Português.
                                   %  Por exemplo, “chapter” irá passar a “capítulo”, “table of contents” a “conteúdo”.
                                   % portuges -- específica para o Português.
\usepackage[utf8]{inputenc} % define o encoding usado texto fonte (input)--usual "utf8" ou "latin1

\usepackage{graphicx} %permite incluir graficos, tabelas, figuras
\usepackage{url} % para utilizar o comando \url{}
\usepackage{enumerate} %permite escolher, nas listas enumeradas, se os iems sao marcados com letras ou numeros-romanos em vez de numeracao normal

%\usepackage{apalike} % gerar biliografia no estilo 'named' (apalike)

\usepackage{color} % Para escrever em cores

\usepackage{multirow} %tabelas com multilinhas
\usepackage{array} %formatação especial de tabelas em array

\usepackage[pdftex]{hyperref} % transformar as referências internas do seu documento em hiper-ligações.

%Exemplos de fontes -- nao e vulgar mudar o tipo de fonte
%\usepackage{tgbonum} % Fonte de letra: TEX Gyre Bonum
%\usepackage{lmodern} % Fonte de letra: Latin Modern Sans Serif
%\usepackage{helvet}  % Fonte de letra: Helvetica
%\usepackage{charter} % Fonte de letra:Charter

\definecolor{saddlebrown}{rgb}{0.55, 0.27, 0.07} % para definir uma nova cor, neste caso 'saddlebrown'

\usepackage{listings}  % para utilizar blocos de texto verbatim no estilo 'listings'
%paramerização mais vulgar dos blocos LISTING - GENERAL
\lstset{
	basicstyle=\small, %o tamanho das fontes que são usadas para o código
	numbers=left, % onde colocar a numeração da linha
	numberstyle=\tiny, %o tamanho das fontes que são usadas para a numeração da linha
	numbersep=5pt, %distancia entre a numeração da linha e o codigo
	breaklines=true, %define quebra automática de linha
    frame=tB,  % caixa a volta do codigo
	mathescape=true, %habilita o modo matemático
	escapeinside={(*@}{@*)} % se escrever isto  aceita tudo o que esta dentro das marcas e nao altera
}
%
%\lstset{ %
%	language=Java,							% choose the language of the code
%	basicstyle=\ttfamily\footnotesize,		% the size of the fonts that are used for the code
%	keywordstyle=\bfseries,					% set the keyword style
%	%numbers=left,							% where to put the line-numbers
%	numberstyle=\scriptsize,				% the size of the fonts that are used for the line-numbers
%	stepnumber=2,							% the step between two line-numbers. If it's 1 each line
%											% will be numbered
%	numbersep=5pt,							% how far the line-numbers are from the code
%	backgroundcolor=\color{white},			% choose the background color. You must add \usepackage{color}
%	showspaces=false,						% show spaces adding particular underscores
%	showstringspaces=false,					% underline spaces within strings
%	showtabs=false,							% show tabs within strings adding particular underscores
%	frame=none,								% adds a frame around the code
%	%abovecaptionskip=-.8em,
%	%belowcaptionskip=.7em,
%	tabsize=2,								% sets default tabsize to 2 spaces
%	captionpos=b,							% sets the caption-position to bottom
%	breaklines=true,						% sets automatic line breaking
%	breakatwhitespace=false,				% sets if automatic breaks should only happen at whitespace
%	title=\lstname,							% show the filename of files included with \lstinputlisting;
%											% also try caption instead of title
%	escapeinside={\%*}{*)},					% if you want to add a comment within your code
%	morekeywords={*,...}					% if you want to add more keywords to the set
%}

\usepackage{xspace} % deteta se a seguir a palavra tem uma palavra ou um sinal de pontuaçao se tiver uma palavra da espaço, se for um sinal de pontuaçao nao da espaço

\parindent=0pt %espaço a deixar para fazer a  indentação da primeira linha após um parágrafo
\parskip=2pt % espaço entre o parágrafo e o texto anterior

\setlength{\oddsidemargin}{-1cm} %espaço entre o texto e a margem
\setlength{\textwidth}{18cm} %Comprimento do texto na pagina
\setlength{\headsep}{-1cm} %espaço entre o texto e o cabeçalho
\setlength{\textheight}{23cm} %altura do texto na pagina

% comando '\def' usado para definir abreviatura (macros)
% o primeiro argumento é o nome do novo comando e o segundo entre chavetas é o texto original, ou sequência de controle, para que expande
\def\darius{\textsf{Darius}\xspace}
\def\antlr{\texttt{AnTLR}\xspace}
\def\pe{\emph{Publicação Eletrónica}\xspace}
\def\titulo#1{\section{#1}}    %no corpo do documento usa-se na forma '\titulo{MEU TITULO}'
\def\super#1{{\em Supervisor: #1}\\ }
\def\area#1{{\em \'{A}rea: #1}\\[0.2cm]}
\def\resumo{\underline{Resumo}:\\ }

%\input{LPgeneralDefintions} %permite ler de um ficheiro de texto externo mais definições

\title{Introdução ao Processamento de Linguagem Natural\\
        Trabalho Prático 3\\
       \textbf{Programa Split-words}\\ Relatório de Desenvolvimento
       } %Titulo do documento
%\title{Um Exemplo de Artigo em \LaTeX}
\author{José André Martins Pereira\\ (a82880@alunos.uminho.pt) \and Ricardo André Gomes Petronilho\\ (a81744@alunos.uminho.pt)
       } %autores do documento
\date{\today} %data

\begin{document} % corpo do documento

\begin{figure}
    \centering
    \includegraphics[scale=0.50]{images/uminho-logo.png}
\end{figure}

\maketitle % apresentar titulo, autor e data

%\tableofcontents % Insere a tabela de indice
%\listoffigures % Insere a tabela de indice figuras
%\listoftables % Insere a tabela de indice tabelas

\section{Contextualização e objetivos}
\hspace{5mm} Na unidade curricular de Introdução ao Processamento de Linguagem Natural, foram-nos propostas várias opções para o trabalho prático três.

\hspace{5mm} Desta forma, a equipa escolheu a opção dois, que consiste no desenvolvimento de um programa, que recebe um texto, onde as palavras se encontram juntas ("coladas"). O objetivo do programa desenvolvido, consiste em determinar as possíveis palavras desse texto, colocando espaços no mesmo.

\section{Solução para o problema}
\hspace{5mm} A solução encontrada pela equipa, para satisfazer os requisitos dos docentes para a opção dois, consiste na construção de \textit{N-Gramas} (N variável), sendo este um treino/conhecimento, feito através de textos, para ser utilizado no programa \textbf{Split-words}. 

\hspace{5mm} A execução do programa necessita sempre deste treino. No entanto, para facilitar o processo, bem como reduzir o tempo de execução, guarda-se o conhecimento em ficheiros, para reutilização.

\hspace{5mm} O tipo de dados utilizado para guardar o treino, consiste num dicionário, onde a chave é a combinação de palavras, e o valor o número de ocorrências da mesma, nos textos utilizados para treino. 

\hspace{5mm} Desta forma, o programa, percorre o dicionário, isto é, as chaves (combinações de palavras) do mesmo, e procura-as no texto recebido, que contém as palavras juntas, substituindo as mesmas (colocando espaços), quando ocorre \textit{match}. Importante referir, como as palavras do texto encontram-se juntas, as chaves do dicionário também são colocadas dessa forma, isto é, juntas, no entanto, numa cópia, mantendo-se o estado original no dicionário. Quando ocorre \textit{match}, visto que se guardou o estado original no dicionário, basta substituir a chave no texto.

\hspace{5mm} No entanto, não se consegue garantir grande precisão, pois, pode existir confusão com determinadas combinações de palavras, como: \textit{"com o", "como"}. Do mesmo modo, a qualidade do treino torna-se muito importante para obtenção de bons resultados, tanto em quantidade, como no tipo de linguagem. A solução possível para resolução do problema referido acima, seria a avaliação das palavras próximas, isto é, a identificação do tipo de palavras (verbos, nomes, determinantes, etc ...), antes e depois da palavra em questão. No entanto, a equipa não teve tempo para implementar esta possível solução. 

\hspace{5mm} Como funcionalidade adicional, ou melhoria de performance, decidiu-se garantir a possibilidade de se aplicar o programa a vários ficheiros. No entanto, tal como referido, foram feitas melhorias de performance a este nível, pois, a execução de vários ficheiros realiza-se em paralelo. Na verdade, são criadas tantas \textit{Threads}, quanto o número de ficheiros, reduzindo-se aproximadamente o tempo em \textbf{y} (y = número de ficheiros) vezes menor.

\section{Observações}
\hspace{5mm} A realização do teste prático do programa deve ser efetuada, lendo previamente o ficheiro \textbf{README}, onde se especifica todas a funcionalidades disponíveis, os argumentos necessários, entre outras notas importantes.

\section{Conclusões}
\hspace{5mm} Em suma, o programa final, não obtém resultados excelentes, tal como referido acima, devido à necessidade de uma melhor avaliação das palavras em questão, no entanto, conclui-se, mesmo assim, que se obtém bons resultados. 

\hspace{5mm}A realização deste trabalho, permitiu, a aplicação dos conhecimentos sobre \textit{N-Gramas}, e uma melhor percepção da utilidade dos mesmos, para diversos problemas. Do mesmo modo, o trabalho contribuiu para o melhor conhecimento da ferramenta \textit{Python} e poder da mesma.

\hspace{5mm} Outros conhecimentos lecionados na unidade curricular, foram aplicados, tais como: expressões regulares, tipos de dados utilizados (dicionário), ordenações, obtenção de argumentos da linha de comandos. 

\hspace{5mm} No entanto, neste trabalho prático, decidiu-se explorar um pouco o paradigma de orientado a objetos, criando-se classes necessárias e respetivas instâncias.

\end{document}
